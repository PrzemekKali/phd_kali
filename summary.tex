\chapter*{Podsumowanie i wnioski}
\addcontentsline{toc}{chapter}{Podsumowanie i wnioski}  
\markboth{Podsumowanie i wnioski}{Podsumowanie i wnioski}
%\section*{Podsumowanie}

W pracy podjęto próbę rozstrzygnięcia wpływu wymiarów poszczególnych elementów konstrukcyjnych na zachowanie dynamiczne łukowych mostów kolejowych. Zwrócono szczególną uwagę na kontekst Kolei Dużych Prędkości. W tym celu pokonano i opisano pełną ścieżkę prowadzącą do uzyskania informacji dających możliwość oceny podatności różnych konfiguracji konstrukcji na drgania przy przejeździe KDP. Pomimo ogromnego zainteresowania naukowców i inżynierów tym tematem od przeszło 50 lat, niektóre złożone konstrukcje wciąż wydają się być pominięte w ocenie przydatności wykorzystania w ciągu linii kolejowych dużych prędkości. Nowoczesne techniki obliczeniowe oraz rozwój technologii sprawiają, że można zaryzykować stwierdzenie, iż każdego typu obiekt da się wybudować w sposób umożliwiający spełnienie wszystkich wymogów nośności, trwałości i bezpiecznego, komfortowego użytkowania. Nie należy jednak pominąć kwestii opłacalności danych rozwiązań, co dla rzeczywistych realizacji jest niezwykle istotne.

Do rozpoczęcia prac nad zagadnieniem dynamiki kolejowych mostów łukowych niezbędne było rozpoznanie i studium literatury dotyczące każdego ze składowych zagadnień. Chcąc urealnić prowadzone prace i umożliwić szeroko pojętą komercjalizację rezultatów, podejście \textit{stricte} naukowe oparte na rozważaniach teoretycznych konfrontowano na każdym kroku z rzeczywistymi systemami konstrukcyjnymi, stosowaną technologią i wymogami zawartymi w obowiązujących przepisach. W pierwszym rozdziale przedstawiono istniejące rozwiązania konstrukcyjne i wybrano do analiz obiekt reprezentatywny w swojej klasie konstrukcji. Przytoczono również aktualną wiedzę dotyczącą oddziaływań dynamicznych na obiekcie kolejowym, aby z pełną świadomością podejść do rozważań badania efektów dynamicznych przejazdu taboru kolejowego po obiekcie inżynieryjnym. Wszystko podsumowano analizą obecnie obowiązujących przepisów. Przytoczono historię rozwoju oraz ocenę niektórych z zapisów. Zwrócono uwagę na potrzebę krytycznego podejścia do kilku z nich, na przykład restrykcyjnego ograniczenia częstotliwości poprzecznych drgań własnych mostów do 1.2 Hz, bez uwzględnienia typu konstrukcji oraz przy braku alternatywnej metody obliczeniowej.

Główny cel pracy osiągnięto przez zastosowanie i połączenie wielu technik numerycznych i badawczych, których wykorzystanie w polskiej inżynierii lądowej - w szczególności mostów - nie jest jeszcze nazbyt popularne. W pracy wdrożono dwa główne narzędzia. Pierwsze to identyfikacja modalna konstrukcji za pomocą Operacyjnej Analizy Modalnej. Drugie to wykorzystanie metaheurystycznych algorytmów optymalizacyjnych przy rozwiązywaniu problemów naukowych i inżynierskich. Wybór obu technik został poprzedzony obszernym studium metod rozwiązywania poszczególnych problemów. Postawiono na sprawdzone metody, których skuteczność udokumentowano w literaturze w praktycznych zastosowaniach. Prace studyjne nad użytymi technikami stanowiły dużą cześć pracy i poświęcono im wiele uwagi. Wynika to z faktu, że niemal wszystkie programy wykorzystane w pracy badawczej zostały od początku do końca stworzone przez autora. Jedynie do tworzenia geometrii, analiz statycznych i dynamicznych wykorzystano oprogramowanie komercyjne MES SOFiSTiK. Autorskie oprogramowanie zawsze musi być sprawdzone na przykładach testowych co również zawarto w pracy. Zdaniem autora umiejętność samodzielnego stworzenia oprogramowania pozwala naukowcowi znacznie szerzej spojrzeć na metody rozwiązywania problemów oraz niejednokrotnie przyspieszyć pracę badawczą.

Identyfikacja modalna z wykorzystaniem Operacyjnej Analizy Modalnej jest obszernie opisana w literaturze, ale jej wykorzystanie w Polsce jest wciąż marginalne. Metoda ta wydaje się być wręcz dedykowana mostom. Są to duże konstrukcje, których wyłączenie z eksploatacji jest kosztowne finansowo i społecznie. Dodatkowo z uwagi na znaczne rozmiary, kontrolowane wzbudzenie mostów (jak w metodach EMA) jest niezwykle trudne. Z przeciwnej strony, narażenie na czynniki środowiskowe oraz ciągły ruch na obiekcie w przypadku OMA są zwykle zaletą. Wybór tej techniki wydaje się więc oczywisty, jeśli potwierdzona jest jej skuteczność w rzeczywistych zastosowaniach. Wśród rodziny OMA wybrano metodę NExT-ERA do identyfikacji modalnej przedmiotowego wiaduktu. Na obiekcie rzeczywistym przeprowadzono badania z wykorzystaniem czujników niskoszumnych. Pomimo mankamentów w przygotowaniu eksperymentu wynikających z braków sprzętowych i ograniczeń administracyjnych, zdaniem autora przeprowadzona identyfikacja zakończyła się w większości powodzeniem. Zidentyfikowano dziesięć pierwszych zestawów parametrów modalnych, co jest liczbą znacznie przekraczającą rzeczywiste zapotrzebowanie w przypadku typowych analiz dynamicznych. Opracowana aplikacja do identyfikacji modalnej został z powodzeniem zastosowana również na innych obiektach mostowych i inżynierskich. Przeprowadzone analizy i badania potwierdziły skuteczność Operacyjnej Analizy Modalnej do identyfikacji mostów. Według autora dobrą praktyką byłoby wdrożenie jej do badań odbiorczych i okresowych. W wielu przypadkach ułatwiłoby i przyspieszyłoby to prace eksperckie. Dodatkowo metoda daje ogromne możliwości wdrożenia jej w systemach monitoringu, z uwagi na możliwość działania w warunkach eksploatacyjnych.

Kolejną z technik użytych do osiągnięcia głównego celu było wykorzystanie metaheurystycznych metod optymalizacji. Okazuje się, że udoskonalone algorytmy oraz wzrost mocy obliczeniowej komputerów osobistych pozwalają na praktycznie wykorzystanie tych metod w rozwiązaniu wielu dotychczas bardzo trudnych lub niesformułowanych analitycznie zagadnień. Wśród metod optymalizacji wybrano prawdopodobnie najpopularniejszą obok algorytmów genetycznych metodę roju cząstek (PSO). W pracy rozwiązano za pomocą tego algorytmu dwa problemy: kalibrację modelu numerycznego oraz poszukiwanie konfiguracji elementów konstrukcyjnych gwarantujących poprawne zachowanie dynamiczne obiektu przy przejeździe Pociągów Dużej Prędkości. Algorytm zgodnie z dotychczasową wiedzą okazał się skuteczny w rozwiązywaniu problemów optymalizacji wielokryterialnej - nawet bardzo złożonych, wielowymiarowych rzeczywistych problemów inżynierskich. Potwierdzono również, że dzięki uniwersalności algorytmów metaheurystycznych początkowe zrozumienie działania metody pozwala później na jej łatwe i szybkie stosowanie przy rozwiązywaniu różnych problemów technicznych. 

Kalibrację konstrukcji mostowych za pomocą PSO prowadzono już w przeszłości. Jednak według najlepszej wiedzy autora, dotychczas nie zastosowano wersji wielokryterialnej tego algorytmu. W pracy rozwiązano problem sformułowany za pomocą 3 funkcji celu oraz 22 zmiennych projektowych. Dzięki kilku kryteriom optymalizacji uzyskano pełen obraz możliwych zmian elementów dopasowania modelu do konstrukcji rzeczywistej związanych ze zidentyfikowanymi częstotliwościami drgań własnych, postaciami drgań oraz przemieszczeniami statycznymi. Metoda ta pozwala elastycznie dopasować się z wyborem ważności poszczególnych funkcji celu do jakości wyników obarczonych nieuniknionymi błędami. W przedmiotowej konstrukcji największe zaufanie posiadano do zidentyfikowanych częstotliwości drgań i głównie na nich oparto wybór najlepszych rozwiązań. Przeprowadzono analizę statystyczną wyników i na tej podstawie określono, z jednej strony, które elementy wpływają istotnie na kalibrację modelu (co odpowiada znanej analizie wrażliwości), a z drugiej wybrano rozwiązanie, które potraktowano jako gwarantujące najlepsze dopasowanie modelu do konstrukcji rzeczywistej. Przeprowadzona analiza pozwala stwierdzić, że zastosowany algorytm umożliwia efektywną kalibrację modelu numerycznego, nawet w przypadku wielu elementów podlegających modyfikacji. Zbudowanie modelu numerycznego poddanego zaawansowanej kalibracji opartej na zidentyfikowanych parametrach było podstawą do prowadzenia dalszych analiz. Wykonane działania zwiększają pewność, że rezultaty nie są poprawne jedynie jakościowo, ale również ilościowo. 

Kolejna część procedury polegała na wykorzystaniu algorytmu PSO do znalezienia zestawu takich rozwiązań konstrukcji, których zastosowanie minimalizuje przyspieszenia pomostu przy przejeździe pociągu i jednocześnie minimalizuje zużycie materiału. Rozpatrzono trzy różne warianty układu wieszaków: proste, ukośne (system Nielsena) i Network. Każdy wariant podzielono na trzy zadania z różnymi prędkościami maksymalnymi przejazdu: od 160 do 300 km/h. W procedurze uwzględniono większość istotnych uwarunkowań normowych.
Dokonano oceny każdego wariantu pod względem odporności dynamicznej i możliwości zastosowania danego układu w przypadku Kolei Dużych Prędkości. Wszystkie przedstawione wyniki dotyczą klasy stalowych obiektów łukowych o rozpiętości około 70 m i nie należy ich uogólniać do innych rozpiętości i układów statycznych. Przeprowadzone analizy pozwoliły na sformułowanie następujących wniosków:
\begin{enumerate}
	
\item W każdym z rozpatrywanych wariantów układu wieszaków i prędkości maksymalnych możliwe jest takie dobranie wymiarów konstrukcji, że spełniony jest warunek maksymalnych przyspieszeń pomostu.
	
\item Udowodniono, że zastosowanie wieszaków typu Nielsena i Network umożliwia lepszą odporność dynamiczną konstrukcji łukowych przy przejazdach pociągów z prędkością do 200 km/h. W tym zakresie prędkości oba układy wykazują dużą odporność dynamiczną i trudne jest takie dobranie wymiarów konstrukcji, które są poprawne statycznie i jednocześnie nie spełniają warunku normowego przyspieszeń pomostu. Obiekt z wieszakami prostymi nie daje takiej gwarancji i koniecznie należy prowadzić analizy dynamiczne dla prędkości przekraczających 160 km/h. 

\item Dla prędkości maksymalnej 300 km/h odpowiednio zaprojektowany wariant z wieszakami prostymi gwarantuje ograniczenie przyspieszeń maksymalnych pomostu do $3.5\;\mathrm{m/s^2}$ dla najmniejszej masy ze wszystkich rozpatrywanych układów wieszaków. Wynika to z faktu, że ten układ wieszaków do obniżenia przyspieszeń preferuje zdecydowanie przyrost sztywności dźwigara/ściągu (przekrój dwuteowy), a nie łuku (przekrój skrzynkowy). Dzieje się to kosztem stopnia wykorzystania naprężeń w analizie statycznej. Dla wieszaków prostych wymagany dynamicznie rozmiar dźwigara jest na tyle duży, że wykorzystanie naprężeń statycznych w dźwigarze nie przekracza 50\%. Warianty z wieszakami ukośnymi i Network umożliwiają bardziej efektywne projektowanie z wykorzystaniem wytrzymałości materiału nawet powyżej 50\%, zwłaszcza dla prędkości niższych niż 300 km/h. Ze względu na wymaganą dużą sztywność elementów konstrukcyjnych podyktowaną względami dynamicznymi, przy tak małym stopniu wytężenia stali S355 być może zasadnym byłoby stosowanie słabszych, tańszych stali.

\item W wariancie z wieszakami prostymi głównym elementem konstrukcyjnym, którego zmiana decyduje o zachowaniu dynamicznym obiektu jest dźwigar/ściąg łuku. W przypadku wieszaków ukośnych i Network, w zależności od rozpatrywanej prędkości elementem decydującym jest łuk lub łuk i ściąg razem. Średnice wieszaków niekiedy wskazują na korelację z wartością maksymalnych przyspieszeń i można stwierdzić, że przyjmują pewne odpowiednie zakresy, które pozwalają na optymalne zapewnienie współpracy łuku i ściągu.

\item Zbyt mała średnica wieszaków może wpływać negatywnie na zachowanie dynamiczne obiektu. Niezależnie od układu wieszaków musi być zapewniona ich wystarczająca sztywność podłużna, aby dźwigar łukowy oraz ściąg współpracowały ze sobą dynamicznie. Podstawowym testem może być analiza modalna. Jeśli w jednej z pierwszych postaci drgań własnych dźwigar łukowy przemieszcza się niemal niezależnie od ściągu to należy zwiększyć średnicę wieszaków. Powyżej granicy związania obu elementów konstrukcyjnych wpływ sztywności ma małe znaczenie dla ogólnego zachowania dynamicznego konstrukcji.

\item Rezultaty optymalizacji przedstawione w przestrzeni dwóch pierwszych modów (częstotliwości i postaci drgań własnych) pokazały związki między postaciami drgań i zmiennymi projektowymi:
\begin{itemize} 
	\item dla wariantu z wieszakami prostymi silną korelację pomiędzy wysokością ściągu i częstotliwością drgań własnych o postaci z dwiema półfalami, oraz wysokością łuku i częstotliwością drgań własnych o postaci z jedną półfalą,
	\item dla wariantu z wieszakami ukośnymi i Network silną korelację pomiędzy wysokością łuku i częstotliwością o postaci drgań z jedną półfalą.
\end{itemize}

\item Dla wariantu wieszaków ukośnych i Network (przy zastosowaniu bardzo dużego  zakresu dla zmiennych projektowych), różnica częstotliwości drgań o postaci z jedną półfalą (od ok. 2.8 do ok. 4 Hz) jest znacznie mniejsza niż częstotliwości drgań o dwóch półfalach (od ok. 3.0 do ok. 8.0 Hz.). Dla wariantu z wieszakami prostymi zmienność częstotliwości o obu postaciach jest porównywalna.

\item Stwierdzono istotny związek pomiędzy dzielnikiem prędkości krytycznej, a maksymalnymi przyspieszeniami pomostu. Do oceny zastosowano przekształcony do częstotliwości wzór na prędkość rezonansową w postaci:
\begin{equation} \label{eq:discussion_critical_freq}
	f_{gr} = \frac{v_{max}\cdot k}{D} 
\end{equation}
gdzie: $v_{max}$ - maksymalna dozwolona prędkość na linii, $k$ - dzielnik prędkości krytycznej, $D$ - rozstaw charakterystyczny oddziaływania kolejowego (np. rozstaw wózków). Przekroczenie tej granicy przez wszystkie zastosowane modele gwarantuje, że w zakresie dopuszczalnych prędkości kolejne osie będą pojawiać się na obiekcie w czasie wielokrotności okresu drgań własnych, a nie w każdym okresie. We wszystkich rozpatrywanych wariantach wieszaków i prędkości koniecznym do uzyskania dopuszczalnych przyspieszeń pomostu było osiągnięcie przez częstotliwość drgań własnych o postaci z podwójną półfalą wartości większej, niż obliczona wg (\ref{eq:discussion_critical_freq}) dla dzielnika $k=1$ i najmniejszej możliwej odległości charakterystycznej modelu $D$. Dla modelu HSLM-A był to dystans $D=18\,\mathrm{m}$. Przekraczanie przez częstotliwość drgań własnych o postaci z podwójną półfalą granic wytyczonych według wzoru (\ref{eq:discussion_critical_freq}) skutkuje znaczącym obniżeniem przyspieszeń maksymalnych pomostu. Powyższa uwaga dotyczy również w mniejszym stopniu częstotliwości o postaci z pojedynczą półfalą. Wartości częstotliwości granicznych są związane ze stosowanymi modelami obciążenia dynamicznego. Opisana powyżej wskazówka powinna byś stosowana wyłącznie dla zestawu pojazdów o podobnej długości całkowitej, jak w przypadku modeli pociągów HSLM-A.




\end{enumerate}

Na podstawie powyższych wniosków można stwierdzić, że postawiona na początku pracy teza została potwierdzona. Możliwe jest określenie wpływu poszczególnych elementów konstrukcyjnych na zachowanie dynamiczne oraz dobór optymalnego rozwiązania gwarantującego najmniejszą możliwą masę konstrukcji przy jednoczesnym założonym minimalnym poziomie przyspieszeń pomostu w trakcie przejazdu taboru kolejowego.

Zdaniem autora wykonane prace mają charakter praktyczny i wdrożeniowy. Szersze zastosowanie identyfikacji dynamicznej obiektów za pomocą Operacyjnej Analizy Modalnej oraz rozwiązywania problemów z wykorzystaniem metaheurystycznych metod optymalizacji wydaje się być uzasadnione z uwagi na zalety obu technik. Przedstawione rezultaty i wnioski mogą również stanowić sugestię dla projektantów co do założeń koncepcyjnych przy projektowaniu nowych mostów w ciągu Linii Dużych Prędkości. Niemniej autor zdaje sobie sprawę z niekompletności opracowania przedstawionego zagadnienia oraz z ograniczeń przedstawionych wniosków. Z tego względu wytyczono następujące prace mające na celu rozwinięcie przeprowadzonych badań i analiz:
\begin{itemize}
	\item Według autora warto wdrożyć i rozpropagować metody Operacyjnej Analizy Modalnej w badaniach odbiorczych i diagnostycznych mostów oraz w systemach monitoringu. Dotyczy to zwłaszcza mostów dużych rozpiętości na których badania tradycyjne są kosztowne lub utrudnione ze względów administracyjnych. Zostaną przeprowadzone prace studyjne nad zastosowaniem tej metody w monitoringu jednego z większych mostów podwieszonych w Polsce.
	\item Autor zamierza kontynuować prace na rozwijaniem oprogramowania do stosowania Operacyjnej Analizy Modalnej. Istnieją nowocześniejsze, odporniejsze na zaszumienie i błędy oraz przyjaźniejsze dla użytkownika metody niż zastosowany algorytm NExT-ERA. Są to miedzy innymi stosowane w oprogramowaniach komercyjnych algorytmy SSI. Jeden z wariantów tej metody został w trakcie pisania zaimplementowany w autorskiej aplikacji.
	\item Godnym rozważenia jest kontynuowanie wykorzystania metod metaheurystycznych do optymalizacji procesów technologicznych oraz innych problemów związanych z mostownictwem. Dzięki zaangażowaniu tego narzędzia możliwe jest rozwiązywanie przeróżnych problemów, które dotychczas były pomijane z uwagi na brak opisu analitycznego, zbyt dużą liczbę zmiennych, bądź zbyt małe możliwości obliczeniowe przeciętnego projektanta lub wykonawcy.
	\item  Algorytmy optymalizacyjne są obecnie rozwijane głównie pod względem wykorzystania uczenia maszynowego w przewidywaniu kolejnych kandydatów, dających największe szanse na bycie optymalnymi. W ramach tej pracy wdrożono jedno z takich podejść, ale nie zostało ono opracowane dostatecznie by mogło zostać bez wątpliwości opublikowane. W najbliższej przyszłości planowane jest udoskonalenie tego rozwiązania.
\end{itemize}  

