
\chapter*{Wprowadzenie}
%\addcontentsline{toc}{chapter}{Wprowdzenie}  
\markboth{Wprowadzenie}{Wprowadzenie}

Linie kolejowe dużych prędkości wciąż są aktualnym elementem procesu rozwoju transportu zarówno w Polsce jak i na świecie. Według UIC w ciągu ostatnich 10 lat globalna liczba pasażerokilometrów z wykorzystaniem kolei dużych prędkości powiększyła się z ok 250 mld pkm/rok do ponad 1000 mld pkm/rok. Prawdą jest, że zdecydowana większość przyrostu wynika z gwałtownego wzrostu inwestycji w KDP w Chinach. Europejskie kraje rozpoczęły eksploatację KDP w latach 80' XX w. W pracy \parencite{Towpik2010} opisano rozwój oraz zalety Kolei Dużych Prędkości. W Polsce z uwagi na warunki polityczne i gospodarcze skonkretyzowane plany budowy LDP zostały przyjęte dopiero w 2008 roku w dokumencie \parencite{UchwaaNr276}. W kolejnej dekadzie postawiono na modernizację istniejących linii kolejowych, w tym na dostosowaniu linii nr 4 (CMK) i linii nr 9 do prędkości 200 km/h. 

\begin{figure}[hbt!]
	\centering
	\includegraphics[width=0.7\linewidth]{/mosty_wstep/mapa.jpg}
	\captionsetup{justification=centering}
	\caption{Mapa planowanych i zrealizowanych linii kolejowych dużych prędkości w Polsce - UIC HIGH-SPEED Congress 2015 \parencite{UIC2015}}
	\label{fig:LDP_mapa}
\end{figure}

Pomimo upływu lat nie ulega wątpliwości, że sieć KDP jest kluczowym elementem rozwoju Państwa \parencite{Raczynski2010}. Pozwala zwiększyć atrakcyjność kolei względem pozostałych środków komunikacyjnych, zwiększają spójność państw wchodzących w sieć połączeń, atrakcyjność gospodarczo-społeczną. W 2017 roku przyjęto kolejną uchwałę mówiącej o rozbudowie Kolei Dużych Prędkości w związku z budową Centralnego Portu Komunikacyjnego \parencite{UchwaaNr173}. 


Oczywiście budowa infrastruktury kolejowej wiąże się przekraczaniem przeszkód i budową obiektów inżynierskich. W przypadku KDP jadący tabor musi mieć zagwarantowane restrykcyjne warunki dotyczące bezpieczeństwa przejazdu i komfortu pasażerów. W konsekwencji, przy projektowaniu mostów związane jest to z większym ograniczeniem przemieszczeń i przyspieszeń konstrukcji \parencite{Niemierko}. Dodatkowo wciąż postępujący rozwój technologi, materiałów budowlanych i informatyzacja procesu projektowania sprawiają, że używane są coraz wytrzymalsze materiały w zoptymalizowanych strukturach. Rośnie także zapotrzebowanie na coraz większe, spektakularne konstrukcje pozwalające na przekroczenie dotychczas niepokonanych przeszkód. Sytuacja ta jest szczególnie widoczna w mostach łukowych, popularnych w ciągu tras kolejowych. Pozwalają one na pokonanie większych rozpiętości niż mosty belkowe, a jednocześnie są konkurencyjne cenowo dla mostów kratownicowych (wer.). Nie bez znaczenia, są również aspekty estetyczne. Odczucia obserwatora mostów odnoszą się zarówno do krajobrazowego znaczenia mostu, jak i do aspektu kulturowego i filozoficznego, które w przypadku mostów łukowych są niezwykle silne \parencite{Kido_Cywiński_2019,Kido_Cywiński_2021}. Mosty łukowe z jednej strony naturalnie odnoszone są do pozytywnie odbieranej tęczy \parencite{Prandowski1994}, a z drugiej w miastach mogą stanowić swoistą bramę wjazdową będąc obiektem typu "landmark". W wyniku tych wszystkich czynników budowane konstrukcje są lżejsze i smuklejsze bądź posiadają nietypowe formy. Z reguły wpływa to na zwiększenie podatności na drgania.

\begin{figure}[hbt!]
	\centering
	\includegraphics[width=0.7\linewidth]{/mosty_wstep/montaz_akvik_sound_bridge.jpg}
	\captionsetup{justification=centering}
	\caption{Montaż szkieletu mostu drogowego o dźwigarze łukowym i wieszakach typu Network}
	\label{fig:bridges_arch_monatage}
\end{figure}

W tradycyjnym podejściu do projektowania mostów kolejowych największy nacisk stawiano na kwestię wytrzymałości ustroju wyrażoną stanem granicznym nośności z uwzględnieniem trwałości zmęczeniowej. Warunki użytkowe - stan graniczny użytkowania - ograniczały się do sprawdzenia największego ugięcia dźwigarów pod obciążeniem normowym \parencite{PKNf}. W stalowych mostach kolejowych zalecano również sprawdzenie okresu poprzecznych drgań własnych przęsła oraz kątów obrotów na podporach. Uwzględnienie naturalnie dynamicznego zjawiska przejazdu taboru po obiekcie ograniczone było jedynie do zastosowania współczynnika dynamicznego, zwiększającego efekty obciążeń statycznych i nieuwzględniającego efektów rezonansu. Należy stwierdzić, że narzucone normami warunki projektowania sprawdziły się w przypadku mostów kolejowych dla kolei poruszającej się z prędkościami poniżej 160 km/h. W nowszych normatywach jakimi sa Eurokody wyraźnie rozdzielono stan graniczny nośności od stanu granicznego zmęczenia, a także istotnie rozbudowano sprawdzenia stanu granicznego użytkowania oraz weryfikację pracy dynamicznej przęsła. Działania te podjęto na podstawie wniosków i doświadczeń z pracy komitetów naukowych badających efekty przejazdów taboru po mostach. W dużej mierze prace te skupione były na przejazdach z prędkością nawet do 350 km/h. Zidentyfikowano i opisano oddziaływania pionowe i poprzeczne występujące wskutek przejazdu. Obecnie w projektowaniu wymagane jest więc nie tylko by konstrukcja była nośna, ale również należy zagwarantować bezpieczeństwo i komfort użytkowania pasażerów. Narzuca to na barki projektanta szereg nowych obowiązkowych sprawdzeń projektowanej konstrukcji. Dynamika typowych konstrukcji belkowych i płytowych przy przejeździe z dużą prędkością została bardzo dobrze rozpoznana i opisana za pomocą tablic i nomogramów. W pozostałych przypadkach wymagane są zaawansowane i czasochłonne testy numeryczne. 

Mostowe konstrukcje łukowe są bardzo różnorodne. Różnią się położeniem pomostu względem dźwigarów, dystrybucją materiału i sztywności między elementami konstrukcyjnymi czy układem elementów łączących dźwigar łukowy z pomostem. Bogactwo form sprawia, że trudno przewidzieć jaki zestaw parametrów będzie optymalny dla danego usytuowania przeprawy i warunków przejazdu. Wszak w poszukiwaniu najlepszego rozwiązania należy uwzględnić nośność konstrukcji, trwałość, bezpieczeństwo, a aktualnie także również komfort dynamiczny przy przejeździe z dużą prędkością. Przy spełnieniu wszystkich obostrzeń rozwiązanie powinno być również możliwie tanie, aby jej wykonanie było opłacalne i miało szansę powstać. Aktualnie mnogość możliwości ukształtowania konstrukcji, sprawia że ostateczne rozwiązanie jest dalekie od optymalnego technicznie jak i finansowo, co może skutkować jego wykluczeniem z realizacji.


\section*{Cel pracy}
Celem pracy jest określenie wpływu poszczególnych elementów konstrukcyjnych i wyznaczenie ich optymalnych parametrów dla stalowego, kolejowego mostu łukowego z jazdą dołem uwzględniając zachowanie dynamiczne obiektu przy przejeździe pociągów dużej prędkości.
W szczególności wykonane na drodze do głównego rezultatu działania miały na celu:
\begin{enumerate}
	\item zweryfikowanie skuteczności metod operacyjnej analizy modalnej przy identyfikacji modalnej stalowych, kolejowych obiektów łukowych średniej rozpiętości,
	\item opracowanie metodyki kalibracji modelu numerycznego mostu kolejowego za pomocą metod wielokryterialnej optymalizacji globalnej, z wykorzystaniem rezultatów identyfikacji modalnej i standardowych pomiarów prowadzonych w trakcie próbnych obciążeń mostów,
	\item opracowanie metodyki optymalizacji wielokryterialnej mostów kolejowych w ciągu LDP ze szczególnym uwzględnieniem dynamiki konstrukcji,
	\item ocena wpływu poszczególnych parametrów konstrukcji mostu łukowego na zachowanie dynamiczne pod obciążeniem pociągami dużej prędkości.
\end{enumerate}


\section*{Teza pracy}
Możliwe jest znalezienie optymalnego finansowo i dynamicznie rozwiązania konstrukcji mostu łukowego w ciągu linii kolejowych dużych prędkości. Rezultaty optymalizacji mogą pozwolić określić zalecenia doboru parametrów konstrukcji łukowej zapewniające minimalny koszt oraz dostateczną odporność dynamiczną na liniach dużych prędkości.

\section*{Struktura pracy}
