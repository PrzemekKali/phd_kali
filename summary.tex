\chapter*{Podsumowanie i wnioski}
\addcontentsline{toc}{chapter}{Podsumowanie i wnioski}  
\markboth{Podsumowanie i wnioski}{Podsumowanie i wnioski}
%\section*{Podsumowanie}

W pracy podjęto próbę określenia wpływu wymiarów poszczególnych elementów konstrukcyjnych na zachowanie dynamiczne określonej klasy łukowych mostów kolejowych. Zwrócono szczególną uwagę na kontekst Kolei Dużych Prędkości. W tym celu przeprowadzono i opisano pełną ścieżkę czynności pozwalających na uzyskanie wyników, dających możliwość oceny podatności na drgania różnych konfiguracji konstrukcji w wyniku przejazdu taboru kolejowego (w tym szczególnie KDP). Pomimo ogromnego zainteresowania badaczy i inżynierów tym tematem od przeszło 50 lat, niektóre złożone konstrukcje wciąż wydają się być pominięte w ocenie wykorzystania w ciągu linii KDP. Można zaryzykować stwierdzenie, że nowoczesne techniki obliczeniowe oraz rozwój technologii budowlanej sprawiają, iż każdego typu obiekt da się zaprojektować i wybudować w sposób umożliwiający spełnienie wszystkich wymogów nośności, trwałości oraz bezpiecznego i komfortowego użytkowania. Jednakże niektóre z nich mogą wymagać znacznie większych nakładów. Dla rzeczywistych realizacji, z punktu widzenia inwestora, jest niezwykle istotne, uwzględnienie w projektowaniu  szeroko pojętych kosztów.

Do rozpoczęcia prac nad zagadnieniem dynamiki kolejowych mostów łukowych niezbędne było szerokie studium literatury. Chcąc urealnić prowadzone prace i umożliwić komercjalizację rezultatów, podejście \textit{stricte} naukowe, oparte na rozważaniach teoretycznych, konfrontowano na każdym kroku z rzeczywistymi systemami konstrukcyjnymi, technologią i wymogami zawartymi w obowiązujących przepisach. W pierwszym rozdziale przedstawiono istniejące rozwiązania konstrukcyjne i wybrano do analiz popularną na drogach kolejowych klasę konstrukcji łukowych: stalowych łuków ze ściągiem i z jazdą dołem. Przytoczono również aktualną wiedzę dotyczącą oddziaływań dynamicznych na obiekcie kolejowym, tak aby z pełną świadomością podejść do modelowania badań efektów dynamicznych przejazdu taboru kolejowego po obiekcie inżynierskim. Całość podsumowano przeglądem obecnie obowiązujących przepisów. Przytoczono historię ich rozwoju oraz ocenę niektórych zapisów. Zwrócono uwagę na potrzebę krytycznego podejścia do kilku z nich, na przykład restrykcyjnego ograniczenia częstotliwości poprzecznych drgań własnych mostów do 1.2 Hz, bez uwzględnienia typu konstrukcji oraz przy braku alternatywnej metody określania ryzyka drgań poprzecznych.

Główny cel pracy osiągnięto przez zastosowanie i połączenie wielu technik numerycznych i badań, których wykorzystanie w polskiej inżynierii lądowej - w szczególności przy badaniu mostów - nie jest powszechne. W pracy wdrożono dwa główne narzędzia. Pierwsze to identyfikacja modalna konstrukcji za pomocą Operacyjnej Analizy Modalnej. Drugie to metaheurystyczne algorytmy optymalizacyjne do rozwiązywania problemów naukowych i inżynierskich. Wybór obu technik został poprzedzony obszernym studium literatury i metod rozwiązywania poszczególnych problemów. Postawiono na sprawdzone algorytmy, których skuteczność została udokumentowana w literaturze w praktycznych zastosowaniach. Studia nad zastosowanymi technikami stanowiły dużą cześć pracy i poświęcono im wiele uwagi. Wynika to z faktu, że niemal wszystkie aplikacje komputerowe wykorzystane w pracy badawczej zostały stworzone przez autora. Jedynie do tworzenia geometrii konstrukcji, analiz statycznych i dynamicznych wykorzystano oprogramowanie komercyjne MES SOFiSTiK. Autorskie oprogramowanie zawsze musi być sprawdzone na przykładach testowych co również pokazano w pracy. Zdaniem autora umiejętność samodzielnego tworzenia oprogramowania pozwala znacznie szerzej spojrzeć na metody rozwiązywania problemów oraz niejednokrotnie przyspieszyć pracę i podnieść jej jakość.

Identyfikacja modalna z wykorzystaniem Operacyjnej Analizy Modalnej jest obszernie opisana w literaturze zagranicznej, ale jej praktyczne stosowanie w Polsce jest rzadkością. Metoda ta wydaje się być wręcz dedykowana mostom. Są to duże konstrukcje, których wyłączenie z eksploatacji jest kosztowne finansowo i społecznie. Dodatkowo, z uwagi na znaczne rozmiary, kontrolowane wzbudzenie mostów (jak w metodach EMA) jest niezwykle trudne. Przeciwnie, dynamiczna odpowiedź konstrukcji na czynniki środowiskowe oraz ciągły ruch na obiekcie w przypadku OMA są zwykle zaletą. Wybór tej techniki wydaje się więc oczywisty, jeśli potwierdzona jest jej skuteczność w rzeczywistych zastosowaniach. Wśród rodziny OMA wybrano metodę NExT-ERA do identyfikacji modalnej analizowanego wiaduktu. Na obiekcie rzeczywistym przeprowadzono badania z wykorzystaniem czujników niskoszumnych. Pomimo mankamentów w przygotowaniu eksperymentu wynikających z braków sprzętowych, doświadczenia i ograniczeń administracyjnych, zdaniem autora przeprowadzona identyfikacja zakończyła się powodzeniem. Zidentyfikowano dziesięć pierwszych zestawów parametrów modalnych, co jest liczbą znacznie przekraczającą rzeczywiste zapotrzebowanie w przypadku typowych analiz dynamicznych. Opracowana aplikacja do identyfikacji modalnej został z powodzeniem zastosowana również na innych obiektach mostowych i inżynierskich. Przeprowadzone analizy i badania potwierdziły skuteczność Operacyjnej Analizy Modalnej do identyfikacji modalnej mostów. Według autora i innych użytkowników tej metody dobrą praktyką byłoby wdrożenie jej do badań odbiorczych i okresowych. W wielu przypadkach ułatwiłoby i przyspieszyłoby to prace eksperckie. Dodatkowo metoda daje ogromne możliwości wdrożenia jej w systemach monitoringu, z uwagi na możliwość działania w warunkach eksploatacyjnych.

Kolejną z technik użytych do osiągnięcia głównego celu było wykorzystanie metaheurystycznych metod optymalizacji. Okazuje się, że udoskonalone algorytmy oraz wzrost mocy obliczeniowej komputerów osobistych pozwalają na praktycznie wykorzystanie tych metod w rozwiązaniu wielu dotychczas bardzo trudnych lub niesformułowanych analitycznie problemów. Wśród metod optymalizacji wybrano, prawdopodobnie najpopularniejszą obok algorytmów genetycznych, metodę roju cząstek (PSO). W pracy rozwiązano za pomocą tego algorytmu dwa problemy: kalibrację wyjściowego modelu numerycznego oraz poszukiwanie wymiarów elementów konstrukcyjnych gwarantujących poprawne zachowanie dynamiczne obiektu przy przejeździe Pociągów Dużej Prędkości. Algorytm zgodnie z dotychczasową wiedzą okazał się skuteczny w rozwiązywaniu zadań optymalizacji wielokryterialnej - nawet bardzo złożonych, wielowymiarowych, rzeczywistych problemów inżynierskich. Potwierdzono również, że zrozumienie metody pozwala później na jej łatwe i szybkie dostosowanie do rozwiązania innych problematycznych zagadnień technicznych. Świadczy to o dużej uniwersalności algorytmów metaheurystycznych.

Kalibrację konstrukcji mostowych za pomocą PSO prowadzono już w przeszłości. Jednak według najlepszej wiedzy autora, dotychczas nie zastosowano wersji wielokryterialnej tego algorytmu. W pracy rozwiązano problem sformułowany za pomocą 3 funkcji celu oraz 22 zmiennych projektowych. Dzięki kilku kryteriom optymalizacji uzyskano pełniejszy obraz możliwego dopasowania modelu do konstrukcji rzeczywistej posługując się zidentyfikowanymi częstotliwościami drgań własnych, postaciami drgań oraz przemieszczeniami statycznymi. W wykonywanych badaniach in situ uznano częstotliwości drgań własnych za element najdokładniej zidentyfikowany i głównie na nich oparto wybór najlepszych rozwiązań. Niemniej w kalibracji uwzględniono również pozostałe funkcje celu związane z postaciami drgań własnych i ugięciami statycznymi. Przeprowadzono analizę otrzymanych rozkładów zmiennych projektowych. Na tej podstawie określono ich wpływ na kalibrację modelu, co w efekcie niejako odpowiada analizie wrażliwości. Ostatecznie otrzymano zbiór bliskich w przestrzeni funkcji celu rozwiązań, z których wybrano jedno do dalszych analiz. Przeprowadzona analiza pozwala stwierdzić, że zastosowany algorytm umożliwia efektywną kalibrację wieloparametrowego modelu numerycznego. Zbudowanie modelu numerycznego poddanego zaawansowanej kalibracji, opartej na zidentyfikowanych parametrach modalnych, gwarantowało przyjęcie rzeczywistych parametrów początkowych i było podstawą do prowadzenia dalszych analiz. Wykonane działania zwiększają pewność, że rezultaty optymalizacji nie są poprawne jedynie jakościowo, ale również ilościowo. 

Kolejna część procedury polegała na wykorzystaniu algorytmu PSO do znalezienia zestawu takich wielkości przekrojów głównych elementów konstrukcji, których zastosowanie minimalizuje przyspieszenia pomostu przy przejeździe pociągu i jednocześnie minimalizuje zużycie materiału. Rozpatrzono trzy różne warianty układu wieszaków: pionowe, ukośne (system Nielsena) i Network (układ siatkowy). Każdy wariant podzielono na trzy odrębne zadania, z różnymi prędkościami maksymalnymi przejazdu: 160, 200 i 300 km/h. W procedurze uwzględniono większość istotnych uwarunkowań normowych (na przykład nie uwzględniono problemu stateczności).
Dokonano oceny każdego wariantu pod względem odpowiedzi dynamicznej i możliwości zastosowania w przypadku Kolei Dużych Prędkości. Wszystkie przedstawione wyniki dotyczą rodziny jednotorowych, stalowych obiektów łukowych z jezdnią na podsypce o rozpiętości około 70 m. W pracy nie dokonano uogólnień do innych rozpiętości i układów statycznych. Przeprowadzone analizy pozwoliły na sformułowanie następujących wniosków:
\begin{enumerate}
	
\item W każdym z rozpatrywanych wariantów układu wieszaków i prędkości maksymalnych możliwe jest takie dobranie wymiarów konstrukcji, że spełniony jest warunek ograniczający maksymalne przyspieszenia pomostu.

\item Przeprowadzone wielokrotnie analizy dynamiczne potwierdziły, że przejazd z maksymalną dopuszczalną prędkością nie musi wywołać największej odpowiedzi konstrukcji. Kluczowe jest wykonanie obliczeń dla przejazdów z prędkościami krytycznymi, dla wszystkich rozpatrywanych modeli obciążenia i kilku pierwszych, istotnych częstotliwości drgań własnych. Zdaniem autora można ograniczyć problem do dwóch pierwszych, \enquote{pionowych} częstotliwości drgań własnych.
	
\item Wykazano, że obiekty łukowe z wieszakami typu Nielsena i Network charakteryzują się korzystną odpowiedzią dynamiczną przy przejazdach pociągów z prędkością do 200 km/h. W tym zakresie prędkości większość wariantów konstrukcji zaprojektowanych na nośność spełnia również kryteria dynamiczne. Obiekt z wieszakami pionowymi zachowuje się znacznie gorzej i koniecznie należy prowadzić analizy dynamiczne dla prędkości przekraczających 160 km/h. 

\item Spośród trzech badanych wariantów układu wieszaków, dla prędkości maksymalnej 300 km/h, odpowiednio zaprojektowany wariant z wieszakami pionowymi osiąga ograniczenie przyspieszeń maksymalnych pomostu do $3.5\;\mathrm{m/s^2}$ dla najmniejszej masy. Wynika to z faktu, że ten układ wieszaków do obniżenia wartości przyspieszeń preferuje zdecydowanie przyrost sztywności dźwigara-ściągu (przekrój dwuteowy), a nie łuku (przekrój skrzynkowy). W przypadku wieszaków ukośnych i Network odpowiedź dynamiczna przęsła zależy zarówno od sztywności ściągu i łuku.

\item Dla prędkości 300 km/h w każdej wersji wieszaków spełnienie warunków dynamicznych wiąże się ze stopniem wykorzystania dopuszczalnych naprężeń. Wymagany ze względu na dynamikę rozmiar dźwigara lub łuku jest na tyle duży, że wykorzystanie naprężeń nie przekracza 50\%. Dla mniejszych prędkości warianty z wieszakami ukośnymi i Network umożliwiają bardziej efektywne projektowanie, z wykorzystaniem wytrzymałości materiału powyżej 50\%. Z uwagi na zastosowaną dużą sztywność elementów konstrukcyjnych podyktowaną wymogami związanymi z dynamiką, przy tak małym stopniu wytężenia stali S355 być może zasadnym byłoby stosowanie gatunków stali o niższych parametrach wytrzymałościowych (na przykład stal S235).

\item W wariancie z wieszakami pionowymi głównym elementem konstrukcyjnym, którego zmiana decyduje o zachowaniu dynamicznym obiektu jest dźwigar-ściąg łuku. 

\item W przypadku wieszaków ukośnych i Network, w zależności od rozpatrywanej prędkości, elementem decydującym jest łuk i średnica wieszaków lub wszystkie elementy razem. 

\item Zbyt mała średnica wieszaków może wpływać negatywnie na zachowanie dynamiczne obiektu. Niezależnie od układu wieszaków musi być zapewniona ich wystarczająca sztywność podłużna, aby dźwigar łukowy oraz ściąg współpracowały ze sobą dynamicznie. Podstawowym testem może być analiza modalna. Jeśli w jednej z pierwszych postaci drgań własnych dźwigar łukowy przemieszcza się niemal niezależnie od ściągu, to należy zwiększyć średnicę wieszaków. 

\item Rezultaty optymalizacji przedstawione w przestrzeni dwóch pierwszych modów (częstotliwości i pionowych postaci drgań własnych) pokazały związki między postaciami drgań i zmiennymi projektowymi. Dokonano następującego uogólnienia dla technicznie realnych wymiarów łuku i ściągu:
\begin{itemize} 
	\item dla wariantu z wieszakami pionowymi stwierdzono silną korelację pomiędzy wysokością ściągu i częstotliwością drgań własnych o postaci z dwiema półfalami, a także pomiędzy wysokością łuku i częstotliwością drgań własnych o postaci z jedną półfalą,
	\item dla wariantu z wieszakami ukośnymi i Network stwierdzono silną korelację pomiędzy wysokością łuku i częstotliwością drgań własnych o postaci z jedną półfalą sinusa, a także pomiędzy średnicą wieszaków i częstotliwością drgań własnych o postaci z podwójną półfalą. Zauważalna, ale nie tak silna korelacja występuje również pomiędzy wysokością ściągu i częstotliwością drgań własnych o postaci z podwójną półfalą. 
\end{itemize}

\item Rozkład rozwiązań w przestrzeni dwóch pierwszych modów pokazał również, jaka może być kolejność występowania modów o postaci z pojedynczą półfalą oraz z podwójną półfalą sinusa. Dla wieszaków pionowych każda z postaci może towarzyszyć pierwszej w kolejności częstotliwości pionowych drgań własnych. W tym przypadku wzajemna kolejność obu postaci jest silnie uzależniona od doboru zmiennych projektowych. W przypadku wieszaków pochylonych (Nielsena i Networku) pomiędzy łukiem i ściągiem powstaje układ quasi kratownicowy. Wśród uzyskanych rozwiązań dla wieszaków w układzie Nielsena, w zdecydowanej większości przypadków pierwsza w kolejności jest częstotliwość pionowych drgań własnych o postaci z pojedynczą półfalą sinusa. Dla wariantu z wieszakami Network we wszystkich otrzymanych rozwiązaniach pierwsza w kolejności jest częstotliwość drgań własnych o postaci z pojedynczą półfalą sinusa.

\item Dla wariantu wieszaków ukośnych i Network zakres częstotliwości drgań własnych o postaci z jedną półfalą wyniósł od ok. 2.8 do ok. 4 Hz. Jest on znacznie mniejszy niż dla zakresu częstotliwości drgań własnych o postaci z dwiema półfalami wynoszącego od ok. 3.0 do ok. 8.0 Hz. Dla wariantu z wieszakami pionowymi rozpiętość zakresów częstotliwości o obu postaciach jest porównywalna. Rezultaty otrzymano przy zastosowaniu bardzo dużego przedziału dopuszczalnego dla zmiennych projektowych i dzięki temu porównanie przedstawionych zakresów jest miarodajne.

\item Stwierdzono istotny związek pomiędzy dzielnikiem prędkości krytycznej, a maksymalnymi przyspieszeniami pomostu. Dzielnik ten wskazuje co który okres drgań kolejne osie obciążenia pojawiają się na obiekcie. Do oceny zastosowano przekształcony do częstotliwości wzór na prędkość krytyczną w postaci:
\begin{equation} \label{eq:discussion_critical_freq}
	f_{gr} = \frac{v_{max}\cdot k}{D} 
\end{equation}
gdzie: $v_{max}$ - maksymalna dozwolona prędkość na linii, $k$ - dzielnik prędkości krytycznej, $D$ - rozstaw charakterystyczny oddziaływania kolejowego (np. rozstaw wózków). Zaprojektowanie konstrukcji, której wybrana częstotliwość drgań własnych jest większa od granicy (\ref{eq:discussion_critical_freq}) dla $k=1$ gwarantuje, że kolejne osie wszystkich zastosowanych modeli obciążenia, poruszających się z prędkością krytyczną, mniejszą niż dopuszczalna prędkość maksymalna, będą pojawiać się na obiekcie w czasie wielokrotności rozpatrywanego okresu drgań własnych, a nie w każdym okresie. We wszystkich uwzględnionych wariantach wieszaków i prędkości, koniecznym do uzyskania dopuszczalnych przyspieszeń pomostu było osiągnięcie przez konstrukcję większej częstotliwości drgań własnych o postaci z podwójną półfalą, niż obliczona wg (\ref{eq:discussion_critical_freq}) dla dzielnika $k=1$ i najmniejszej możliwej odległości charakterystycznej modelu $D$. Dla modelu HSLM-A był to rozstaw wózków $D=18\,\mathrm{m}$. Przekraczanie przez częstotliwość drgań własnych o postaci z podwójną półfalą granic wytyczonych według wzoru (\ref{eq:discussion_critical_freq}) skutkuje znaczącym obniżeniem przyspieszeń maksymalnych pomostu. Powyższa uwaga dotyczy również w mniejszym stopniu częstotliwości drgań własnych o postaci z pojedynczą półfalą. 
% Wartości częstotliwości granicznych, ze względu na parametr $D$, związane są ze stosowanymi modelami obciążenia dynamicznego. Opisana powyżej ocena jest wykonana wyłącznie dla zestawu modeli pociągów HSLM-A. %Zmienna długość modeli obciążenia sprawia, że przy danej prędkości całkowity czas przejazdu przez konstrukcję jest zróżnicowany. W takiej sytuacji powyższe wskazówki powinny być stosowane ostrożnie.

\end{enumerate}

Na podstawie powyższych wniosków można stwierdzić, że postawiona na początku pracy teza została potwierdzona. Możliwe jest określenie wpływu poszczególnych elementów konstrukcyjnych na zachowanie dynamiczne oraz dobór optymalnego rozwiązania gwarantującego najmniejszą możliwą masę konstrukcji przy jednoczesnym spełnieniu kryteriów dynamicznych w trakcie przejazdu taboru kolejowego.

Zdaniem autora wykonane analizy i ich rezultaty mają szansę na wdrożenie w przypadku projektowania i badania rzeczywistych obiektów mostowych. Szersze zastosowanie identyfikacji dynamicznej obiektów za pomocą Operacyjnej Analizy Modalnej oraz rozwiązywanie problemów z wykorzystaniem metaheurystycznych metod optymalizacji wydaje się być uzasadnione z uwagi na przedstawione w pracy zalety obu technik. Uzyskane rezultaty i wnioski mogą również stanowić sugestię dla projektantów dotyczącą założeń koncepcyjnych przy projektowaniu nowych mostów w ciągu Linii Dużych Prędkości. Autor proponuje następujący, podstawowy algorytm postępowania przy projektowaniu wiaduktu łukowego na Linii Dużych Prędkości:
\begin{enumerate}

\item Określenie parametrów wyjściowych do projektu oraz warunków eksploatacyjnych (głównie prędkości maksymalnych).

\item Dobór parametrów konstrukcji spełniających statyczne warunki Stanu Granicznego Nośności i Stanu Granicznego Użytkowania.

\item Analiza modalna i określenie częstotliwości granicznych dla modeli HSLM-A według wzoru (\ref{eq:discussion_critical_freq}).

\item Porównanie \enquote{pionowych} częstotliwości drgań własnych z częstotliwościami granicznymi. Jeżeli \enquote{pionowa} częstotliwość drgań o postaci z podwójną półfalą jest mniejsza niż wszystkie częstotliwości graniczne dla $k=1$, to należy zmienić wymiary konstrukcji w celu podniesienia częstotliwości drgań własnych z podwójną półfalą.

\item Kompleksowe analizy dynamiczne przeprowadzić dopiero jeżeli warunki z punktu 4 są spełnione.

\end{enumerate}
\noindent
W trakcie projektowania warto wykorzystać następujące zależności i relacje:
\begin{itemize}
	\item podwyższenie częstotliwości drgań własnych z postacią o podwójnej półfali najefektywniej osiąga się przez: 
	\begin{itemize}
	\item zwiększenie sztywności giętnej ściągu w przypadku wieszaków prostych, 
	\item zwiększenie sztywności ściągu lub sztywności podłużnej wieszaków w przypadku wieszaków Nielsena i Network,
	\end{itemize}
	\item podwyższenie częstotliwości drgań własnych z postacią o pojedynczą półfalą najefektywniej osiąga się przez zwiększenie sztywności łuku,
	\item dla prędkości przejazdu do 200 km/h, układy wieszaków Nielsena i Network zapewniają mniejszą odpowiedź dynamiczną niż wieszaki proste,
	\item analizy dynamiczne na etapie działań projektowych, zaleca się ograniczyć do prędkości krytycznych i prędkości maksymalnej. Zwiększenie zakresu obliczeń do pełnego, zalecanego normą zestawu prędkości warto zastosować dla ostatecznego rozwiązania projektowego.
\end{itemize}


Autor zdaje sobie sprawę z niekompletności opracowania przedstawionego zagadnienia oraz z ograniczeń sformułowanych wniosków. Z tego względu wytyczono następujące prace mające na celu rozwinięcie przeprowadzonych badań i analiz:
\begin{itemize}
	\item Według autora warto wdrożyć i rozpropagować metody Operacyjnej Analizy Modalnej w badaniach odbiorczych i diagnostycznych mostów oraz w systemach monitoringu. Dotyczy to zwłaszcza mostów dużych rozpiętości, na których badania tradycyjne są kosztowne lub utrudnione ze względów administracyjnych. Autor planuje przeprowadzenie prac studyjnych nad zastosowaniem tej metody w monitoringu jednego z większych mostów podwieszonych w Polsce.
	\item Autor planuje kontynuować prace na rozwijaniem oprogramowania do stosowania Operacyjnej Analizy Modalnej. Istnieją nowocześniejsze, odporniejsze na zaszumienie i błędy oraz przyjaźniejsze dla użytkownika metody niż zastosowany algorytm NExT-ERA. Są to miedzy innymi zaimplementowane w programach komercyjnych algorytmy SSI. Jeden z wariantów tej metody został zaimplementowany w autorskiej aplikacji i jest w fazie testów.
	\item Godnym rozważenia jest kontynuowanie wykorzystania metod metaheurystycznych do optymalizacji procesów technologicznych oraz innych problemów związanych z mostownictwem. Dzięki zaangażowaniu tego narzędzia możliwe jest rozwiązywanie przeróżnych problemów, które dotychczas były pomijane z uwagi na brak opisu analitycznego, zbyt dużą liczbę zmiennych, bądź zbyt małe możliwości obliczeniowe przeciętnego projektanta lub wykonawcy.
	\item  Algorytmy optymalizacyjne są obecnie rozwijane głównie pod względem wykorzystania uczenia maszynowego w przewidywaniu kolejnych kandydatów, dających największe szanse na bycie optymalnymi rozwiązaniami. W ramach tej pracy wdrożono jedno z takich podejść, ale nie zostało ono dopracowane dostatecznie dobrze, aby mogło zostać bez wątpliwości opublikowane. W najbliższej przyszłości planowane jest udoskonalenie tego rozwiązania.
\end{itemize}  




