\begin{figure}
	\subfloat{\includegraphics[width=6.5cm, valign=t,trim=80 200 80 200, clip]{pg_w_i_l_i_s_kolor-01.eps}} \hfill
	\subfloat{\includegraphics[height=1.5cm, valign=t]{Wydział_WILiŚ_CMYK.pdf}}
	
\end{figure}


\begin{myfont}
	\thispagestyle{plain}	
	\noindent
	\\ \\	
	\textbf{OPIS ROZPRAWY DOKTORSKIEJ}
	
	\footnotesize \noindent
	\\ \\
	\textbf{Autor rozprawy doktorskiej:} mgr inż. Przemysław Kalitowski
	
	\noindent
	\\
	\textbf{Tytuł rozprawy doktorskiej w języku polskim:} Przęsło łukowe wiaduktu kolejowego. Wpływ schematu statycznego i rozwiązań konstrukcyjnych na własności dynamiczne.
	
	\noindent
	\\
	\textbf{Tytuł rozprawy w języku angielskim:} A railway arch bridge. An influence of a static scheme and a construction solution to dynamic properties.
	
	\noindent
	\\
	\textbf{Język rozprawy doktorskiej:} polski
	
	\noindent
	\\
	\textbf{Promotor rozprawy doktorskiej:} dr hab. inż. Krzysztof Żółtowski, prof. uczelni
	
	\noindent
	\\
	{\bfseries \sout{Drugi promotor rozprawy doktorskiej*:}}
	
	\noindent
	\\
	{\bfseries \sout{Promotor pomocniczy rozprawy doktorskiej*}}
	
	\noindent
	\\
	{\bfseries \sout{Kopromotor rozprawy doktorskiej*:}}
	
	\noindent
	\\
	\textbf{Data obrony:}
	
	\noindent
	\\
	\textbf{Słowa kluczowe rozprawy doktorskiej w języku polskim:} most łukowy ze ściągiem, kolej dużych prędkości, operacyjna analiza modalna, optymalizacja rojem cząstek, analiza dynamiczna
	
	\noindent
	\\
	\textbf{Słowa kluczowe rozprawy doktorskiej w języku angielskim:} tied-arch bridge, high-speed rail, operational modal analysis, particle swarm optimization, dynamic analysis 
	\vfill
	
	\pagebreak[4]
	
	\thispagestyle{plain}
	\noindent
	\\
	\textbf{Streszczenie rozprawy w języku polskim:}\\
	W pracy przedstawiono kompleksową analizę optymalizacyjną wiaduktu kolejowego. Szczególną uwagę zwrócono na zachowanie dynamiczne pod obciążeniem kolejowym. Odpowiedź dynamiczna przęsła pod obciążeniem kolei dużych prędkości jest jednym z kluczowych elementów podlegających ocenie przy projektowaniu konstrukcji. Jest ograniczana ze względu na trwałość konstrukcji oraz komfort pasażerów. Sformułowano procedurę, w której połączono analizę numeryczną Metodą Elementów Skończonych oraz wielokryterialną optymalizację rojem cząstek (PSO) do poszukiwania zestawu najlepszych rozwiązań ze względu na \higr{masę wykonanej} konstrukcji i odpowiedź dynamiczną. Procedurę zastosowano na przykładzie istniejącego łukowego wiaduktu kolejowego WK-2 o rozpiętości 70 m w ciągu Pomorskiej Kolei Metropolitalnej. Parametry dynamiczne obiektu zidentyfikowano za pomocą procedury NExT-ERA Operacyjnej Analizy Modalnej. Zbudowany model MES \higg{skalibrowano} wykorzystując wyniki badań i metodę optymalizacji PSO. Przeanalizowano dziewięć wariantów optymalizacji różniących się układem wieszaków oraz maksymalną prędkością przejazdu. Uwzględniono wieszaki proste, ukośne oraz Network i prędkości 160, 200 i 300 km/h. Zestawione wyniki pozwoliły na ocenę przydatności i opłacalności poszczególnych wariantów konstrukcji w zastosowaniach w ciągu kolei dużych prędkości. Podjęto próbę znalezienia zależności pomiędzy wielkością poszczególnych elementów konstrukcyjnych, a odpowiedzią dynamiczną wiaduktu.
	
	\noindent
	\\
	\textbf{Streszczenie rozprawy w języku angielskim:}\\
	The paper presents a comprehensive optimization analysis of the railway viaduct. Particular attention was paid to the dynamic performance under railway load. The dynamic response of the span under the load of high-speed rail is one of the crucial elements to be assessed in the structure's design. It is limited because of the durability of the structure and passenger comfort. A procedure was formulated in which the numerical analysis with the Finite Element Method and multi-objective optimization with a Particle Swarm Optimization (PSO) were combined. The aim was to search for a set of the best solutions in terms of the cost of construction and dynamic response. The case study was conducted on the example of the existing 70 m long, tied-arch railway viaduct, along the Pomeranian Metropolitan Railroad. The dynamic parameters of the structure were identified using the NExT-ERA technique of the Operational Modal Analysis. The constructed FEM model was calibrated using the test results. The PSO method was also used for calibration. Nine optimization variants were analyzed, differing in the arrangement of hangers and the maximum speed of passage. Straight, Nielsen and Network types of hangers and speeds of 160, 200 and 300 km/h were considered. The summarized results allowed to evaluate the usefulness and profitability of particular design variants in high-speed rail applications. An attempt was made to find the relationship between the stiffness of individual structural elements and the dynamic response of the viaduct.
	
	\noindent
	\\
	{\bfseries \sout{Streszczenie rozprawy w języku, w którym została napisana**:}}
	
	\noindent
	\\
	{\bfseries \sout{Słowa kluczowe rozprawy doktorskiej w języku, w którym została napisana**:}}
	\bigskip
	
	*) niepotrzebne skreślić.
	
	**) dotyczy rozpraw doktorskich napisanych w innych językach, niż polski lub angielski.
	\vfill
	\pagebreak[4]
\end{myfont}
