\chapter{Optymalizacja metodą roju cząstek - Particle Swarm Optimizaton}
\section*{Wprowadzenie}
Znalezienie najlepszej możliwej konfiguracji elementów konstrukcyjnych, zapewniającej poprawnie przeniesienie obciążeń statycznych, zapewniającej komfort dynamiczny i najlepiej możliwie taniej jest zadaniem, które na co dzień towarzyszy projektantom mostów. Takie zadanie może kojarzyć się intuicyjnie z pojęciem optymalizacji, czyli wyborem najlepszego z wielu rozwiązań, pozwalającego osiągnąć cel lub cele. \cite{Szymczak1995} określa następuje elementy, jakie powinno zawierać poprawnie sformułowane zadanie optymalizacji projektu
\begin{itemize}[noitemsep]
	\item kryteria optymalizacji - miarę spełnienia danego celu,
	\item parametry optymalizacji - parametry systemu, które są stałe lub niezależne od projektanta, 
	\item zmienne projektowe - parametrów systemu zależne od projektanta,
	\item ograniczenia - elementy określające zakres dopuszczalnych rozwiązań. 
\end{itemize}
Kryterium optymalizacji powinno w sposób wymierny pozwolić na ocenę danego rozwiązania. Podstawowymi kryteriami stosowanymi w przypadku konstrukcji może być koszt jej wykonania, ilość materiału czy nakład pracy. Kryterium, które decyduje o wyborze najlepszego rozwiązania nazywane jest funkcją celu. Parametry i zmienne projektowe 


\cite{Tesch2016} zaproponował klasyfikację metod optymalizacji zależnie od następujących parametrów:
\begin{itemize}[noitemsep]
	\item asdas
	\begin{itemize}
		\item asdas
	\end{itemize}
\end{itemize}
Dzięki informatyzacji i komputerom możliwe jest zastosowanie optymalizacji w procesach o dużej złożoności i trudności projektowanych obiektów. 
\section{Meta-heurystyczne metody optymalizacji}
\section{Particle Swarm optimization}
\subsection{Opis metody}
\subsection{Zastosowania}
\subsection{Przykład teoretyczny}