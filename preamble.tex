%preamble\vect{\psi}_i^{T}\vect{K\psi}_j = 0\vect{\psi}_i^{T}\vect{K\psi}_j = 0
%https://www.overleaf.com/learn/latex/How_to_Write_a_Thesis_in_LaTeX_(Part_4):_Bibliographies_with_BibLaTeX
%\documentclass[12pt,twoside]{report}
\documentclass[12pt,twoside,openany]{book}
\usepackage[a4paper,width=150mm,top=25mm,bottom=25mm,bindingoffset=10mm]{geometry}
\pdfminorversion=7



\usepackage{polski}
\usepackage[utf8]{inputenc}
\usepackage[T1]{fontenc}
\usepackage{lmodern}
\usepackage[polish]{babel}
\usepackage[autostyle]{csquotes}
 

%spis tresci
%\usepackage{tocloft}
%\setlength{\cftbeforetoctitleskip}{0pt}

\usepackage[svgnames,table]{xcolor}
\usepackage{amsfonts}
\usepackage{arial}    
\newenvironment{myfont}{\fontfamily{ua1}\selectfont}{}
\usepackage[normalem]{ulem}


%\usepackage{layouts}
\DeclareUnicodeCharacter{2208}{$\epsilon$}
\DeclareUnicodeCharacter{0229}{\k{e}} %zmiana na ę po eksporcie z mendeleya
\DeclareUnicodeCharacter{2264}{$\ge$}
\DeclareUnicodeCharacter{2265}{$\le$}
\DeclareUnicodeCharacter{0327}{\k{a}} 
%kiedy bibliografia jest gotowa:
% - trzeba zmienic w pliku bib bo wstawia "aą". z \c{a} na \k{a}.
% - podkreslniki z \{_} na _
\DeclareUnicodeCharacter{2212}{–}
\DeclareUnicodeCharacter{221A}{$\sqrt{\cdot}$}
\DeclareUnicodeCharacter{FFFD}{\textbf{?}}
%BIBLIOGRAFIA
%\usepackage[backend=biber,style=authoryear,url=false]{biblatex}
\usepackage{ifthen}
\usepackage[
backend=biber,
citestyle=numeric-comp,
style=numeric,
sorting=none,
giveninits=true,
url=true
]{biblatex}
\DeclareNameAlias{default}{family-given}

\DeclareFieldFormat{url}{%
	\mkbibacro{URL}\addcolon\space
	\ifhyperref
	{\href{#1}{\nolinkurl{\thefield{urlraw}}}}
	{\expandafter\nolinkurl\expandafter{\romannumeral-`0\thefield{urlraw}}}
}

\AtEveryBibitem{
	\ifthenelse{\ifentrytype{misc}}
	{}    
	{ 	\clearfield{url}%
		\clearlist{language}%
		\clearfield{month}%
		\clearfield{day}%
	}}

%\AtEveryBibitem{%
%	\clearlist{language}%
%	\clearfield{month}%
%}

%style=numeric-comp,
\addbibresource{2020_08_19_My_Collection.bib}
\AtBeginBibliography{\small}

%NAGLOWEK I STOPKA
%\pagestyle{headings}
\usepackage{fancyhdr}
\pagestyle{fancy}
\fancyhf{}
\fancyhead[ro]{\slshape\nouppercase{\rightmark}}
\fancyhead[le]{\slshape\nouppercase{\leftmark}}
\fancyfoot[CE,CO]{\thepage}
%\setlength{\headheight}{14.49998pt}
\setlength{\headheight}{27.12pt}

%OBRAZKI
\usepackage{pgfplots}
\usepackage{graphicx}
\graphicspath{ {./figures/} }
\usepackage{caption}
\captionsetup[figure]{font=small,justification=centering,singlelinecheck=false}
\captionsetup[table]{font=small,justification=justified,singlelinecheck=false}
\usepackage[export]{adjustbox}
%\usepackage{subfig}
\usepackage{subcaption}
\usepackage{float}
\newsavebox{\measurebox}


%wzory
\usepackage{amsmath,bm}
\usepackage{mathtools}
\usepackage{scalefnt}


\newcommand\ScaleExists[1]{\vcenter{\hbox{\scalefont{#1}$\exists$}}}
\DeclareMathOperator*\bigexists{%
	\vphantom\sum
	\mathchoice{\ScaleExists{1.5}}{\ScaleExists{1.4}}{\ScaleExists{1}}{\ScaleExists{0.75}}}
\newcommand\ScaleForall[1]{\vcenter{\hbox{\scalefont{#1}$\forall$}}}
\DeclareMathOperator*\bigforall{%
	\vphantom\sum
	\mathchoice{\ScaleForall{1.5}}{\ScaleForall{1.4}}{\ScaleForall{1}}{\ScaleForall{0.75}}}


%lista
\usepackage{enumitem}

%tabele
\usepackage{booktabs}
\usepackage{ragged2e}
\usepackage[tableposition=above]{caption}
\usepackage{pifont}
\usepackage{longtable}
\usepackage{multirow}
\usepackage{array}
\newcolumntype{P}[1]{>{\centering\arraybackslash}p{#1}}
\newcolumntype{Y}{>{\centering\arraybackslash}X}
\usepackage{tabularx}

%w tabelach, grubsza linia
\makeatletter
\def\thickhline{%
	\noalign{\ifnum0=`}\fi\hrule \@height \thickarrayrulewidth \futurelet
	\reserved@a\@xthickhline}
\def\@xthickhline{\ifx\reserved@a\thickhline
	\vskip\doublerulesep
	\vskip-\thickarrayrulewidth
	\fi
	\ifnum0=`{\fi}}
\makeatother

\newlength{\thickarrayrulewidth}
\setlength{\thickarrayrulewidth}{2\arrayrulewidth}




%\usepackage{xcolor}
%wlasne polecenia
\newcommand{\teng}[1]{(\textit{eng. #1})}
\newcommand{\vect}[1]{\bm{#1}}
\newcommand{\matr}[1]{\mathbf{#1}}



%definicje
\usepackage{amsthm}
\newtheoremstyle{break}
	{\topsep}{\topsep}%
	{}{}%
	{\bfseries}{}%
	{\newline}{}%

\theoremstyle{break}
\newtheorem{definition}{Definicja}[]
\newtheorem{theorem}{Twierdzenie}[]

%linki w pracy
%\usepackage[hidelinks]{hyperref}

%strona pusta
\usepackage{afterpage}
\newcommand\myemptypage{
	\null
	\thispagestyle{empty}
	\vspace{7cm}
	
	\begin{center}
		\textit{PUSTA STRONA}
	\end{center}

	\vfill
	%\addtocounter{page}{-1}
	\newpage
}

%Highlighty
\usepackage{color}
\usepackage{soulutf8}
\newcommand{\higr}[1]{\sethlcolor{pink}\hl{#1}}
\newcommand{\higg}[1]{\sethlcolor{green}\hl{#1}}

%to do notes
%\usepackage[textwidth=20mm]{todonotes}
\usepackage{todonotes}
\setlength{\marginparwidth}{2cm}


%end of preamble