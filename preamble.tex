%preamble\vect{\psi}_i^{T}\vect{K\psi}_j = 0\vect{\psi}_i^{T}\vect{K\psi}_j = 0
%https://www.overleaf.com/learn/latex/How_to_Write_a_Thesis_in_LaTeX_(Part_4):_Bibliographies_with_BibLaTeX
\documentclass[12pt,twoside]{report}
\usepackage[a4paper,width=150mm,top=25mm,bottom=25mm,bindingoffset=6mm]{geometry}

\usepackage{polski}
\usepackage[utf8]{inputenc}
\usepackage[]{fontenc}
\usepackage[polish]{babel}
\usepackage[autostyle]{csquotes} 

\usepackage[svgnames,table]{xcolor}
\usepackage{amsfonts}

%\usepackage{layouts}
\DeclareUnicodeCharacter{2208}{$\epsilon$}
\DeclareUnicodeCharacter{0229}{ę}
\DeclareUnicodeCharacter{2264}{$\ge$}
\DeclareUnicodeCharacter{0327}{?????}

%BIBLIOGRAFIA
\usepackage[backend=biber,style=authoryear,url=false]{biblatex}
\addbibresource{2020_08_19_My_Collection.bib}
\AtBeginBibliography{\small}

%NAGLOWEK I STOPKA
%\pagestyle{headings}
\usepackage{fancyhdr}
\pagestyle{fancy}
\fancyhf{}
\fancyhead[ro]{\slshape\nouppercase{\rightmark}}
\fancyhead[le]{\slshape\nouppercase{\leftmark}}
\fancyfoot[CE,CO]{\thepage}
%OBRAZKI
\usepackage{pgfplots}
\usepackage{graphicx}
\graphicspath{ {./figures/} }
\usepackage{caption}
\usepackage[export]{adjustbox}
\usepackage{subfig}
\usepackage{float}
\newsavebox{\measurebox}

%wzory
\usepackage{amsmath,bm}
\usepackage{mathtools}
\usepackage{scalefnt}


\newcommand\ScaleExists[1]{\vcenter{\hbox{\scalefont{#1}$\exists$}}}
\DeclareMathOperator*\bigexists{%
	\vphantom\sum
	\mathchoice{\ScaleExists{1.5}}{\ScaleExists{1.4}}{\ScaleExists{1}}{\ScaleExists{0.75}}}
\newcommand\ScaleForall[1]{\vcenter{\hbox{\scalefont{#1}$\forall$}}}
\DeclareMathOperator*\bigforall{%
	\vphantom\sum
	\mathchoice{\ScaleForall{1.5}}{\ScaleForall{1.4}}{\ScaleForall{1}}{\ScaleForall{0.75}}}


%lista
\usepackage{enumitem}

%tabele
\usepackage{booktabs}
\usepackage{ragged2e}
\usepackage[tableposition=above]{caption}
\usepackage{pifont}
\usepackage{longtable}
\usepackage{multirow}

\newcommand{\teng}[1]{(\textit{eng. #1})}
\newcommand{\vect}[1]{\bm{#1}}
\newcommand{\matr}[1]{\mathbf{#1}}

%definicje
\usepackage{amsthm}
\newtheoremstyle{break}
	{\topsep}{\topsep}%
	{}{}%
	{\bfseries}{}%
	{\newline}{}%

\theoremstyle{break}
\newtheorem{definition}{Definicja}[]

%end of preamble