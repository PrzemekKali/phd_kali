\chapter*{Zastosowane skróty}
%\begin{itemize}[label = {}]
%	\item \textbf{OMA}
%	\item \textbf{OMA}
%	\item LDP
%	\item KDP
%	\item ASD
%	\item FFT
%\end{itemize}
%
%% Please add the following required packages to your document preamble:
%% \usepackage{booktabs}
\makeatletter
\setlength{\@fptop}{0pt}
\makeatother


\begin{table}[h]
	\footnotesize
	\setlength\extrarowheight{5pt}
	\begin{tabular}{P{0.1\textwidth} P{0.4\textwidth} P{0.4\textwidth}}
		\toprule
		\textbf{Akromin} & \textbf{Znaczenie EN} & \textbf{Znaczenie PL} 					\\ \midrule
		EMA     & Experimental Modal Analysis         & Eksperymentalna Analiza Modalna         \\ \midrule
		ERA		& Eigensystem Realization Algorithm & Algorytm Realizacji Własnej \\ \midrule
		ERRI	& European Railway Research Institute & Europejski Instytut Kolejnictwa \\ \midrule
		HSLM	& High Speed Load Model (HSLM) & - \\ \midrule
		KDP		& -									&Kolej Dużych Prędkości 				\\  \midrule
		LDP		& -									&Linia Dużych Prędkości 		\\ \midrule
		NExT 	& Natural Excitation Technique & - \\ \midrule
		MES		& Finite Element Method	& Metoda Elementów Skończonych \\ \midrule
		MRS		& Finite Difference Method & Metoda Różnic Skończonych \\ \midrule
		OMA     & Operational Modal Analysis         & Operacyjna Analiza Modalna   \\  \midrule
		PDP		& -									&Pociąg Dużych Prędkości 		\\ \midrule
		PSO		& Particle Swarm Optimization	& Optymalizacja Rojem Cząstek \\ \midrule
		TSI		& Technical Specifications of the Interoperability &Warunki Techniczne Interoperacyjności \\ \midrule
		\multirow{2}{*}{UIC}		& International Union of Railways 	& \multirow{2}{*}{Międzynarodowy Związek Kolei}\\
		& (fr. Union Intenationale des Chemins de fer)& \\ \midrule
	\end{tabular}
\end{table}


\chapter*{Słownik podstawowych pojęć}
\begin{itemize}[label = {},leftmargin=*]
\item \textbf{Landmark} - budowla wyróżniająca się pod względem widokowym i będąca rozpoznawalnym symbolem miejsca, w którym się znajduje.
\item \textbf{Kalibracja} (modelu) - działania mające na celu taką modyfikację modelu numerycznego, aby jego przewidywania w danym zakresie były jak najbardziej zgodne z odpowiednikiem rzeczywistym.
\item \textbf{Walidacja} - ogół czynności mających na celu zbadanie odpowiedniości, trafności lub dokładności czegoś.
\item \textbf{Optymalizacja} - 
\item \textbf{Algorytm metaheurystyczny} -
\item \textbf{Wysokość konstrukcyjna} - największa odległość pomiędzy dolną krawędzią przęsła (z uwzględnieniem wyposażenia mostu i instalacji zewnętrznych), a niweletą.
\item \textbf{Dyskretyzacja} - proces przekształcający opis pola wyrażony za pomocą nieskończenie wielu parametrów w opis wyrażony przez skończoną liczbę wartości zlokalizowanych w skończonej liczbie punktów.
\end{itemize}