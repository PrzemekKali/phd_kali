\thispagestyle{plain}
\section*{Zastosowane skróty}
\makeatletter
\setlength{\@fptop}{0pt}
\makeatother
\begin{table}[H]
	\footnotesize
	\setlength\extrarowheight{5pt}
	\begin{tabular}{Z{0.1\textwidth} Z{0.4\textwidth} Z{0.4\textwidth}}
		\toprule
		\textbf{Akronim} & \textbf{Znaczenie EN} & \textbf{Znaczenie PL} 					\\ \midrule
		CFST		& Concrete Filled Steel Tubes	& Rury Stalowe Wypełnione Betonem \\
		DFT		& Discrete Fourier Transformation	& Dyskretna Transformata Fouriera \\ %\midrule
		EMA     & Experimental Modal Analysis         & Eksperymentalna Analiza Modalna         \\ %\midrule
		ERA		& Eigensystem Realization Algorithm & Algorytm Realizacji Własnej \\ %\midrule
		ERRI	& European Railway Research Institute & Europejski Instytut Kolejnictwa \\ %\midrule
		FDM		& Frequency-Domain Method & Metody w dziedzinie częstotliwości \\
		FRF		& Frequency Response Function	& Funkcja Odpowiedzi Częstotliwościowej \\ %\midrule
		FFT		& Fast Fourier Transformation	& Szybkie Przekształcenie Fouriera \\ %\midrule
		HSLM	& High Speed Load Model (HSLM) & - \\ %\midrule
		IRF		& Inpulse Response Function & Funkcja Odpowiedzi Impulsowej \\ %\midrule
		KDP		& High-Speed Rail		&Kolej Dużych Prędkości 				\\  %\midrule
		LDP		& High-Speed Lines							&Linia Dużych Prędkości 		\\ %\midrule
		LDT		& Logarithmic Decrement		& Logarytmiczny Dekrement Tłumienia \\ %\midrule
		MAC		& Modal Assurance Criterion & - \\
		MDOF	& Multiple Degree of Freedom & Wiele Stopni Swobody \\ %\midrule
		MOPSO	& Multi-Objective Particle Swarm Optimization & Wielokryterialna Optymalizacja Rojem Cząstek \\
		MPC		& Modal Phase Colinearity & - \\
		NExT 	& Natural Excitation Technique & - \\ %\midrule
		MES		& Finite Element Method	& Metoda Elementów Skończonych \\ %\midrule
		MRS		& Finite Difference Method & Metoda Różnic Skończonych \\ %\midrule
		OMA     & Operational Modal Analysis         & Operacyjna Analiza Modalna   \\  %\midrule
		PDP		& High-Speed Train									&Pociąg Dużych Prędkości 		\\ %\midrule
		PKM		& - 								&Pomorska Kolej Metropolitalna	\\ %\midrule
		PSD		& Power Spectral Density	& Widmowa Gęstość Mocy \\
		PSO		& Particle Swarm Optimization	& Optymalizacja Rojem Cząstek \\ %\midrule
		RMS		& Root Mean Square	& Średnia kwadratowa (wartość skuteczna) \\
		SDOF	& Single Degree of Freedom	& Jeden Stopień Swobbody \\ %\midrule
		SSI		& Stochastic Subspace Identification & - \\
		TDM		& Time-Domain Method & Metody w dziedzinie czasu \\ 
		TSI		& Technical Specifications of the Interoperability &Warunki Techniczne Interoperacyjności \\ %\midrule
		\multirow{2}{*}{UIC}		& International Union of Railways 	& \multirow{2}{*}{Międzynarodowy Związek Kolei}\\
		& (fr. Union Internationale des Chemins de fer)& \\ %\midrule
		WPD	& Dynamic Amplification Factor & Współczynnik Przewyższenia Dynamicznego \\
	\end{tabular}
\end{table}
\vfill
\pagebreak[4]

\thispagestyle{plain}
\section*{Słownik podstawowych pojęć}
\begin{itemize}[label = {},leftmargin=0pt]
\item \textbf{Algorytm metaheurystyczny} - algorytm heurystyczny wyższego poziomu, rozwiązujący problem - tak jak pierwowzór - na bazie zgromadzonych doświadczeń, ale cechujący się sformułowaniem niepodporządkowanym konkretnemu typowi problemu i wykorzystaniem randomizacji. 
\item \textbf{Dyskretyzacja} - proces przekształcający opis pola wyrażony za pomocą nieskończenie wielu parametrów w opis wyrażony przez skończoną liczbę wartości zlokalizowanych w skończonej liczbie punktów.
\item \textbf{Kalibracja} (modelu) - działania mające na celu taką modyfikację modelu numerycznego, aby jego przewidywania w danym zakresie były jak najbardziej zgodne z odpowiednikiem rzeczywistym.
\item \textbf{Kolej Dużych Prędkości} - podsystem kolejowych przewozów pasażerskich cechujący się istotnie większą prędkością handlową pociągów niż inne rodzaje przewozów. Według Międzynarodowego Związku Kolei (UIC) o kolei dużych prędkości mówi się gdy tabor osiąga prędkość powyżej 250 km/h.
\item \textbf{Landmark} - budowla wyróżniająca się pod względem wizualnym i będąca rozpoznawalnym symbolem miejsca, w którym się znajduje.
\item \textbf{Optymalizacja} - poszukiwanie za pomocą metod matematycznych najlepszego, ze względu na wybrane kryterium, rozwiązania danego zagadnienia, przy uwzględnieniu określonych ograniczeń.
\item \textbf{Pasażerokilometr} - (pkm) jednostka stosowana w transporcie pasażerskim, będąca iloczynem liczby pasażerów na danej trasie i długości tej trasy.
\item \textbf{Solver} - oprogramowanie komputerowe umożliwiające rozwiązywanie równań lub układów równań.
\item \textbf{Walidacja} - ogół czynności mających na celu zbadanie odpowiedniości, trafności lub dokładności czegoś.
\item \textbf{Wysokość konstrukcyjna} - największa odległość pomiędzy dolną krawędzią przęsła (z uwzględnieniem wyposażenia mostu i instalacji zewnętrznych), a niweletą.

\end{itemize}
\vfill
\pagebreak[4]

