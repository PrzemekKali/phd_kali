%	
%\begin{titlepage}
%	\begin{center}
%		\includegraphics[width=\textwidth]{logotyp PG i WILiŚ.pdf}
%		\vspace{3cm}
%		
%		\Large
%		\textbf{Rozprawa doktorska\\}
%		\vspace{1cm}
%		\LARGE
%		\textbf{Przęsło łukowe wiaduktu kolejowego.\\Wpływ schematu statycznego i rozwiązań konstrukcyjnych na własności dynamiczne.}
%		
%		\vspace{2cm}
%		\large
%		Autor\\mgr inż. Przemysław Kalitowski \\
%		\vspace{0.5cm}
%		Promotor\\dr hab. inż. Krzysztof Żółtowski, prof. uczelni
%		\vfill
%		
%
%		\large
%		
%		Politechnika Gdańska\\
%		Wydział Inżynierii Lądowej i Środowiska\\
%		Katedra Transportu Szynowego i Mostów
%		
%		\vspace{1cm}
%		Gdańsk, lipiec 2021
%		
%	\end{center}
%\end{titlepage}

%\thispagestyle{empty}
\begin{titlepage}
	%	\begin{center}
	\begin{figure}
		\subfloat{\includegraphics[width=6.5cm, valign=t,trim=80 200 80 200, clip]{pg_w_i_l_i_s_kolor-01.eps}}\hfill
		\subfloat{\includegraphics[height=1.5cm, valign=t]{Wydział_WILiŚ_CMYK.pdf}}
	\end{figure}
	
	\begin{myfont}
		\vspace*{0cm}
		
		\footnotesize \noindent
		Imię i nazwisko autora rozprawy: mgr inż. Przemysław Kalitowski
		
		\noindent
		Dyscyplina naukowa: inżynieria lądowa i transport
		
		\vspace{3.5cm}
		
		\normalsize \noindent
		\textbf{ROZPRAWA DOKTORSKA}
		
		\vspace{3.5cm}
		
		\footnotesize \noindent%
		Tytuł rozprawy w języku polskim: Przęsło łukowe kolejowego obiektu mostowego. Wpływ \mbox{schematu} statycznego i rozwiązań konstrukcyjnych na właściwości dynamiczne.%
		
		\vspace{0.5cm}
		
		\noindent%
		Tytuł rozprawy w języku angielskim: A railway arch bridge. An influence of a static scheme and a construction solution to dynamic properties.
		
		%\vspace{2cm}		
		\vfill
		
		\begin{table}[H]
			\footnotesize
			\begin{myfont}
				\renewcommand{\arraystretch}{2}
				\begin{tabular}{|p{0.47\textwidth}|p{0.47\textwidth}|}
					\hline
					\begin{tabular}[c]{@{}l@{}}Promotor   \\[1cm]   podpis\end{tabular}  & \begin{tabular}[c]{@{}l@{}}Drugi promotor \\[1cm] podpis\end{tabular} \\ \hline
					dr hab. inż. Krzysztof Żółtowski, prof. uczelni   & --  \\ \hline
					\begin{tabular}[c]{@{}l@{}}Promotor   pomocniczy    \\[1cm]  podpis\end{tabular} & \begin{tabular}[c]{@{}l@{}}Kopromotor \\[1cm]  podpis\end{tabular}  \\ \hline
					--   &  --  \\ \hline
				\end{tabular}
			\end{myfont}
		\end{table}
		\noindent
		\begin{center}
			Gdańsk, rok 2022
		\end{center}
	\end{myfont}
	%	\end{center}
\end{titlepage}